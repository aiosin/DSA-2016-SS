% LaTeX Template f�r Datenstrukturen und Algorithmen Abgaben
% Autor: Sandro Speth
% Bei Fragen: Sandro.Speth@studi.informatik.uni-stuttgart.de
\documentclass[12pt]{article}
\usepackage[latin1]{inputenc}
\usepackage[T1]{fontenc}
\usepackage[ngerman]{babel}
\usepackage{graphicx}
\usepackage{color}
\usepackage{listings}
\usepackage[a4paper,lmargin={2cm},rmargin={2cm},tmargin={3.5cm},bmargin = {2.5cm},headheight = {4cm}]{geometry}
\usepackage{amsmath,amssymb,amstext}
\usepackage{amsthm}
\usepackage[lined,algonl,boxed]{algorithm2e}
\usepackage{tikz}
\usepackage[inline]{enumitem}
\usepackage{fancyhdr}
\usepackage{hyperref}
\pagestyle{fancy} 
\fancyhf{}

\renewcommand{\theenumi}{(\alph{enumi})}
\renewcommand{\labelenumi}{\text{\theenumi}}

\newcounter{sheetnr}
\setcounter{sheetnr}{07} % Nummer des �bungsblattes

\newcounter{exnum}
\setcounter{exnum}{1} % Nummer der Aufgabe

\newcommand{\aufgabe}[1]{\section*{Aufgabe \theexnum\stepcounter{exnum}: #1}} % Befehl f�r Aufgabentitel

% Rechter Teil der Kopfzeile:
% Namen und Matrikelnummern aller Bearbeiter
\rhead{Wilhelm Buchm�ller (3133783)\\
Daniel Wanner (3149308)\\
Artur Frenzen (2736424)}

% Linker Teil der Kopfzeile
\lhead{Datenstrukturen \& Algorithmen\\
Sommersemester 2016\\
�bungsblatt \thesheetnr}

% Beginn des eigentlichen Dokuments
\begin{document}
% Aufgabe 1
\aufgabe{Tiefensuche}
\begin{itemize}
	
\item
	E \(\rightarrow\) D \(\rightarrow\) G \(\rightarrow\) C \(\rightarrow\) B \(\rightarrow\) A \(\rightarrow\) F
\item

	B \(\rightarrow\) A \(\rightarrow\) F \(\rightarrow\) G \(\rightarrow\) C \(\rightarrow\) E \(\rightarrow\) D


\end{itemize}


% Aufgabe 2
\aufgabe{Topologisches Sortieren}
% Teilaufgaben
\begin{enumerate}
	\item 
	Die topologische Sortierung gibt eine der m�glichen zul�ssigen Sortierungen eines gerichteten Graphen G zur Abarbeitung der Knoten an.
	Beispielsweise eine Sortierung von zu erledigenden Aufgaben in Reihenfolge.
	
	\item Nein, die Gewichtung einer Kante spielt bei der Sortierung nur die Richtung einer Kante spielt eine Rolle f�r die Topologische Sortierung.
	\item
	Nein, denn es kann mehrere g�ltige Sortierungen geben. Das liegt daran, dass freie Kanten zuf�llig gew�hlt werden.
	
	\item
	\begin{itemize}
		\item
		\begin{tikzpicture}
		\begin{scope}[every node/.style={circle,thick,draw}]
    \node (1) at (0,0) {A};
    \node (2) at (3,0) {B};
    \node (3) at (0,-2) {C};
    \node (4) at (3,-2) {D};
	\node (5) at (0,-4) {E};
	\node (6) at (3,-4) {F};
	\node (7) at (1.5,-5) {G};
    
    
    
	\end{scope}

	\begin{scope}[>={stealth[black]},
              every node/.style={fill=white,circle},
              every edge/.style={draw=red,very thick}]
              
    \path [->] (2) edge (1);
    \path [->] (1) edge [bend right=30](5);
    \path [->] (2) edge (5);
    
    \path [->] (2) edge [bend left=90] (7);
    
	\path [->] (6) edge (7);    
    
    \path [->] (5) edge (6);
    

	\path [->] (7) edge [bend left=20] (3);
	\path [->] (4) edge (7);
	\path [->] (6) edge (3);
	\path [->] (5) edge (4);
		
    
\end{scope}
\end{tikzpicture}

		\item
		\begin{itemize}
		
		
			\item Ausgangsknoten A
			\item Besuch von A
			\item Schritt zu E
			\item Besuch von E
			\item Schritt zu D
			\item Besuch von D
			\item Schritt zu G
			\item Besuch von G
			\item Schritt zu C
			\item Besuch von C
			\item Einf�rben von C
			\item Einf�rben von G
			\item Einf�rben von D
			\item Schritt (von E) zu F
			\item Besuch von F
			\item Schritt zu C
			\item Schritt (von F) zu G
			\item Einf�rben von F
			\item Einf�rben von E
			\item Einf�rben von A
			\item Besuch von B
			\item Schritt (von B) zu A
			\item Schritt zu E
			\item Schritt zu G
			\item Einf�rben von B
		
			
		\end{itemize}
		Es ergibt sich eine m�gliche Priority-Queue von \(G_s\): \big[ B, A, E, F, D, G, C \big]
	\end{itemize}
	\item
	Ja, solche Graphen existieren. Eine topologische Sortierung funktioniert nicht bei 				zyklischen Graphen.
\end{enumerate}


% Aufgabe 3
\aufgabe{\framebox[1.1\width]{Impl} in Java}

\paragraph{}\noindent\rule{8cm}{0.4pt}


\aufgabe{Euler, Hamilton und k�rzeste Wege}
\begin{enumerate}
	\item
	\begin{itemize}
		\item Ja. Ein Pfad hei�t Euler`scher Weg, wenn jede Kante des Graphen genau einmal in seinem Pfad vorkommt: C, B, A, C, D, F, E, D, G, A, C
		\item Nein, da in einem Euler`schen Kreis alle Knoten einen geraden Grad haben. Dieser Graph besitzt jedoch zwei Knoten die einen ungeraden Grad haben: E und C.
		\item Ja. Ein Pfad hei�t Hamilton`scher Weg, wenn er alle Knoten eines Graphen genau einmal durchl�uft: C, B, A, G, D, F, E
	\end{itemize}
	\item Der Bellman-Ford-Algorithmus.
Begr�ndung: Da er als einziger Algorithmus Wege mit negativen Kantengewichten erkennt. Hier vorhanden von e nach c (-1).

\end{enumerate}
% Ende des Dokuments
\end{document}
