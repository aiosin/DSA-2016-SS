% LaTeX Template f�r Datenstrukturen und Algorithmen Abgaben
% Autor: Sandro Speth
% Bei Fragen: Sandro.Speth@studi.informatik.uni-stuttgart.de
\documentclass[12pt]{article}
\usepackage[latin1]{inputenc}
\usepackage[T1]{fontenc}
\usepackage[ngerman]{babel}
\usepackage{graphicx}
\usepackage{color}
\usepackage{listings}
\usepackage[a4paper,lmargin={2cm},rmargin={2cm},tmargin={3.5cm},bmargin = {2.5cm},headheight = {4cm}]{geometry}
\usepackage{amsmath,amssymb,amstext}
\usepackage{amsthm}
\usepackage[lined,algonl,boxed]{algorithm2e}
\usepackage{tikz}
\usepackage[inline]{enumitem}
\usepackage{fancyhdr}
\pagestyle{fancy} 
\fancyhf{}

\renewcommand{\theenumi}{(\alph{enumi})}
\renewcommand{\labelenumi}{\text{\theenumi}}

\newcounter{sheetnr}
\setcounter{sheetnr}{8} % Nummer des �bungsblattes

\newcounter{exnum}
\setcounter{exnum}{1} % Nummer der Aufgabe

\newcommand{\aufgabe}[1]{\section*{Aufgabe \theexnum\stepcounter{exnum}: #1}} % Befehl f�r Aufgabentitel

% Rechter Teil der Kopfzeile:
% Namen und Matrikelnummern aller Bearbeiter
\rhead{Wilhelm Buchm�ller(3133783)\\
Daniel Wanner(3149308)\\
Artur Frenzen(2736424)}

% Linker Teil der Kopfzeile
\lhead{Datenstrukturen \& Algorithmen\\
Sommersemester 2016\\
�bungsblatt \thesheetnr}

% Beginn des eigentlichen Dokuments
\begin{document}
% Aufgabe 2
\aufgabe{Dijkstra-Algorithmus}
Platzhalter

% Aufgabe 2
\aufgabe{Delaunay-Triangulierung}
% Teilaufgaben
\begin{enumerate}
	\item 
	\item 
\end{enumerate}


% Aufgabe 3
\aufgabe{\framebox[1.1\width]{Impl} in Java}

\paragraph{}\noindent\rule{8cm}{0.4pt}

\aufgabe{Algorithmus von Kruskal}
\aufgabe{Literaturrecherche}
\begin{enumerate}
	\item
	\begin{itemize}
		\item E.W. Dijkstra gibt bei dem erstem Problem zwei Schritte die wiederholt werden bis das Problem gel�st ist
		\item E.W. Dijkstra gibt bei dem zweiten Problem zwei Anmerkungen an
	\end{itemize}
	\item 
	\begin{verbatim}
@article{
    dijkstra1959note,
    title={A note on two problems in connexion with graphs},
    author={Dijkstra, Edsger W},
    journal={Numerische mathematik},
    volume={1},
    number={1},
    pages={269--271},
    year={1959},
    publisher={Springer}
}

    \end{verbatim}
	\item
	\begin{verbatim}
@Article{Dijkstra1959,
    author="Dijkstra, E. W.",
    title="A note on two problems in connexion with graphs",
    journal="Numerische Mathematik",
    year="1959",
    volume="1",
    number="1",
    pages="269--271",
    issn="0945-3245",
    doi="10.1007/BF01386390",
    url="http://dx.doi.org/10.1007/BF01386390"
}

	\end{verbatim}
	\item Da es verschiedene BibTEX-repositories gibt, unter anderem auch eine von Google-Scholar und vom Springer-Verlag hat jede Platform eine eigene Style-Convention 
\end{enumerate}
% Ende des Dokuments
\end{document}
