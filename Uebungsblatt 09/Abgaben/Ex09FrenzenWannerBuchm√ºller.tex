% LaTeX Template für Datenstrukturen und Algorithmen Abgaben
% Autor: Sandro Speth
% Bei Fragen: Sandro.Speth@studi.informatik.uni-stuttgart.de
\documentclass[12pt]{article}
\usepackage[latin1]{inputenc}
\usepackage[T1]{fontenc}
\usepackage[german]{babel}
\usepackage{graphicx}
\usepackage{color}
\usepackage{listings}
\usepackage[a4paper,lmargin={2cm},rmargin={2cm},tmargin={3.5cm},bmargin = {2.5cm},headheight = {4cm}]{geometry}
\usepackage{amsmath,amssymb,amstext}
\usepackage{amsthm}
\usepackage[lined,algonl,boxed]{algorithm2e}
\usepackage{tikz}
\usepackage[inline]{enumitem}
\usepackage{fancyhdr}



\pagestyle{fancy}
\fancyhf{}

\renewcommand{\theenumi}{(\alph{enumi})}
\renewcommand{\labelenumi}{\text{\theenumi}}

\newcounter{sheetnr}
\setcounter{sheetnr}{9} % Nummer des Übungsblattes

\newcounter{exnum}
\setcounter{exnum}{1} % Nummer der Aufgabe

\newcommand{\aufgabe}[1]{\section*{Aufgabe \theexnum\stepcounter{exnum}: #1}} % Befehl für Aufgabentitel

% Rechter Teil der Kopfzeile:
% Namen und Matrikelnummern aller Bearbeiter
\rhead{Wilhelm Buchm"uller(3133783)\\
Artur Frenzen (2736424)\\
Daniel Wanner (3149308)}

% Linker Teil der Kopfzeile
\lhead{Datenstrukturen \& Algorithmen\\
Sommersemester 2016\\
Uebungsblatt \thesheetnr}

% Beginn des eigentlichen Dokuments
\begin{document}
% Aufgabe 2
\aufgabe{KMP-Pr"afixtabelle}

\begin{itemize}

\item 
Knuth-Morris-Pratt Pr"afixtabelle
\begin{figure}[h]
\begin{tabular}{|l|l|l|l|l|l|l|l|l|l|l|l|}
\hline
Position k im Pattern            & 0 & 1 & 2 & 3 & 4 & 5 & 6 & 7 & 8 & 9 & 10 \\ \hline
Pattern p{[}k{]}                 & d & u & m & d & i & d & u & m & b & u & m  \\ \hline
L"angster pr"afix{[}k{]}; f{[}k{]} & 0 & 0 & 0 & 1 & 0 & 1 & 2 & 3 & 0 & 0 & 0  \\ \hline
\end{tabular}
\end{figure}

\item 
\begin{enumerate}
\item
Mismatch an index 4 \newline pr"afix ist 1 also starten wir bei d
\begin{figure}[h]
\begin{tabular}{llllllllllllllllllllllllll}
\hline
d & u & m & d & u & m & d & i & d & i & d & u & m & d & i & d & u & m & b & u & m & d & i & b & u & m \\ \hline
  &   &   &   & x &   &   &   &   &   &   &   &   &   &   &   &   &   &   &   &   &   &   &   &   &   \\
d & u & m & d & i & d & u & m & b & u & m &   &   &   &   &   &   &   &   &   &   &   &   &   &   &   \\ \hline
\end{tabular}
\end{figure}
\item
Hier haben wir Mismatches am ersten Index von Index 5 bis Index 7, keine Pr"afix Verschiebung, wir starten wie gewohnt bei d
\begin{figure}[h]
\begin{tabular}{llllllllllllllllllllllllll}
\hline
d & u & m & d & u & m & d & i & d & i & d & u & m & d & i & d & u & m & b & u & m & d & i & b & u & m \\ \hline
  &   &   &   & x &   &   &   &   &   &   &   &   &   &   &   &   &   &   &   &   &   &   &   &   &   \\
  &   &   &   & d & u & m & d & i & d & u & m & b & u & m &   &   &   &   &   &   &   &   &   &   &   \\ \hline
\end{tabular}

\end{figure}

\item Mismatch an Index 8 bei Pattern-Index 1 

\begin{figure}[!h]
\begin{tabular}{llllllllllllllllllllllllll}
\hline
d & u & m & d & u & m & d & i & d & i & d & u & m & d & i & d & u & m & b & u & m & d & i & b & u & m \\ \hline
  &   &   &   &   &   &   & x  &   &   &   &   &   &   &   &   &   &   &   &   &   &   &   &   &   &   \\
  &   &   &   &   &   & d & u & m & d & i & d & u & m & b & u & m &   &   &   &   &   &   &   &   &   \\ \hline
\end{tabular}

\end{figure}

\item Mismatch an Index 10
\begin{figure}[!h]
\begin{tabular}{llllllllllllllllllllllllll}
\hline
d & u & m & d & u & m & d & i & d & i & d & u & m & d & i & d & u & m & b & u & m & d & i & b & u & m \\ \hline
  &   &   &   &   &   &   &   &   & x &   &   &   &   &   &   &   &   &   &   &   &   &   &   &   &   \\
  &   &   &   &   &   &   &   &   & d & u & m & d & i & d & u & m & b & u & m &   &   &   &   &   &   \\ \hline
\end{tabular}

\end{figure}

\item Patten gefunden
\begin{figure}[!h]
\begin{tabular}{llllllllllllllllllllllllll}
\hline
d & u & m & d & u & m & d & i & d & i & d & u & m & d & i & d & u & m & b & u & m & d & i & b & u & m \\ \hline
  &   &   &   &   &   &   &   &   &   &   &   &   &   &   &   &   &   &   &   &   &   &   &   &   &   \\
  &   &   &   &   &   &   &   &   &   & d & u & m & d & i & d & u & m & b & u & m &   &   &   &   &   \\ \hline
\end{tabular}
\end{figure}

\end{enumerate}


\end{itemize}

% Aufgabe 2
\aufgabe{Boyer-Moore}
\paragraph{}\noindent\rule{8cm}{0.4pt}


% Aufgabe 3
\aufgabe{Regul"are Ausdr"ucke}
\begin{itemize}
\item \begin{enumerate}
\item Ausschliesslich nach per Skript erlaubten Regeln\newline 
\verb/(^.......a$)/ 

\item Nach ''erweiterten''-Regeln \newline

\verb/(^.{7}a$/
\end{enumerate} 
\item \verb/(^(Schnell|Fischer|Patroullien)(boot+)(e?)$)/
\item \verb/(^.+)(Apfel|Birne)(.+$)/
\item \verb/(^((Y.*|N.*)(a+)(y))|((Mua)(ha+))(.*)(s)$)/
\end{itemize}





% Ende des Dokuments
\end{document}
