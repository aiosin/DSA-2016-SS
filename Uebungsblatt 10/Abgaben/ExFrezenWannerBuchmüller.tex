% LaTeX Template für Datenstrukturen und Algorithmen Abgaben
% Autor: Sandro Speth
% Bei Fragen: Sandro.Speth@studi.informatik.uni-stuttgart.de
\documentclass[12pt]{article}
\usepackage[latin1]{inputenc}
\usepackage[T1]{fontenc}
\usepackage[ngerman]{babel}
\usepackage{graphicx}
\usepackage{color}
\usepackage{listings}
\usepackage[a4paper,lmargin={2cm},rmargin={2cm},tmargin={3.5cm},bmargin = {2.5cm},headheight = {4cm}]{geometry}
\usepackage{amsmath,amssymb,amstext}
\usepackage{amsthm}
\usepackage[lined,algonl,boxed]{algorithm2e}
\usepackage{tikz}
\usepackage[inline]{enumitem}
\usepackage{fancyhdr}
\pagestyle{fancy} 
\fancyhf{}

\renewcommand{\theenumi}{(\alph{enumi})}
\renewcommand{\labelenumi}{\text{\theenumi}}

\newcounter{sheetnr}
\setcounter{sheetnr}{09} % Nummer des Übungsblattes

\newcounter{exnum}
\setcounter{exnum}{1} % Nummer der Aufgabe

\newcommand{\aufgabe}[1]{\section*{Aufgabe \theexnum\stepcounter{exnum}: #1}} % Befehl für Aufgabentitel

% Rechter Teil der Kopfzeile:
% Namen und Matrikelnummern aller Bearbeiter
\rhead{Max Mustermann (1234567)\\
Klaus Kleber (1234568)\\
Melanie Marshmallow (1234569)}

% Linker Teil der Kopfzeile
\lhead{Datenstrukturen \& Algorithmen\\
Sommersemester 2016\\
Ubungsblatt \thesheetnr}

% Beginn des eigentlichen Dokuments
\begin{document}
% Aufgabe 2
\aufgabe{Textalgorithmen/Levenshtein-Distanz}
\begin{enumerate}
\item \text{}\\ 
\begin{figure}[!h]
\begin{tabular}{|l|l|l|l|l|l|l|l|l|l|}
\hline
  & $\epsilon$  & t & r & e & k & k & i & e & s \\ \hline
$\epsilon$  & 0 & 1 & 2 & 3 & 4 & 5 & 6 & 7 & 8 \\ \hline
w & 1 & 1 & 2 & 3 & 4 & 5 & 6 & 7 & 8 \\ \hline
a & 2 & 2 & 2 & 3 & 4 & 5 & 6 & 7 & 8 \\ \hline
r & 3 & 3 & 5 & 3 & 4 & 5 & 6 & 7 & 8 \\ \hline
s & 4 & 4 & 3 & 3 & 4 & 5 & 6 & 7 & 7 \\ \hline
\end{tabular}
\end{figure}
\item Die Levenshtein-Distanz zwischen zwei Worten befindet sich immer in der untersten rechten Ecke.
\end{enumerate}
% Aufgabe 2
\aufgabe{Hashfunktionen}
% Teilaufgaben
\begin{enumerate}
\item A
\item B
\item C
\item D
\end{enumerate}


% Aufgabe 3
\aufgabe{Hashing}

\begin{enumerate}
	\item PLATZHALTER
	\item PLATZHALTER
	\item PLATZHALTER
	
\end{enumerate}

% Bildumgebung


% Ende des Dokuments
\end{document}
