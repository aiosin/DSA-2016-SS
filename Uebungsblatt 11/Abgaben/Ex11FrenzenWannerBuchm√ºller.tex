% LaTeX Template f�r Datenstrukturen und Algorithmen Abgaben
% Autor: Sandro Speth
% Bei Fragen: Sandro.Speth@studi.informatik.uni-stuttgart.de
\documentclass[12pt]{article}
\usepackage[latin1]{inputenc}
\usepackage[T1]{fontenc}
\usepackage[ngerman]{babel}
\usepackage{graphicx}
\usepackage{color}
\usepackage{listings}
\usepackage[a4paper,lmargin={2cm},rmargin={2cm},tmargin={3.5cm},bmargin = {2.5cm},headheight = {4cm}]{geometry}
\usepackage{amsmath,amssymb,amstext}
\usepackage{amsthm}
\usepackage[lined,algonl,boxed]{algorithm2e}
\usepackage{tikz}
\usepackage[inline]{enumitem}
\usepackage{fancyhdr}
\pagestyle{fancy} 
\fancyhf{}

\renewcommand{\theenumi}{(\alph{enumi})}
\renewcommand{\labelenumi}{\text{\theenumi}}

\newcounter{sheetnr}
\setcounter{sheetnr}{09} % Nummer des �bungsblattes

\newcounter{exnum}
\setcounter{exnum}{1} % Nummer der Aufgabe

\newcommand{\aufgabe}[1]{\section*{Aufgabe \theexnum\stepcounter{exnum}: #1}} % Befehl f�r Aufgabentitel

% Rechter Teil der Kopfzeile:
% Namen und Matrikelnummern aller Bearbeiter
\rhead{ Wilhelm Buchm�ller  (3133783)\\
Daniel Wanner ()\\
Melanie Marshmallow (1234569)}

% Linker Teil der Kopfzeile
\lhead{Datenstrukturen \& Algorithmen\\
Sommersemester 2016\\
�bungsblatt \thesheetnr}

% Beginn des eigentlichen Dokuments
\begin{document}
% Aufgabe 2
\aufgabe{Petrinetze}
-------
% Aufgabe 2
\aufgabe{Das Raucherproblem}
\begin{itemize}
	\item -
	\item Es ensteht eine Art "`Lock"' zwischen Raucher 1 und Raucher 2. Der dritte Raucher kommt nie zum Rauchen da die erstem beiden Raucher sich die Zutaten immer wegschnappen.

	\item Man k�nnte praktisch eine Art Priority-Queue verwenden, oder ein zuf�lligen Raucher w�hlen.
	
	\
	
\end{itemize}



% Aufgabe 3
\aufgabe{Regular Expresssions}

\begin{enumerate}
	\item Impl in Java
	\item -

\end{enumerate}

% Bildumgebung


% Ende des Dokuments
\end{document}
